\documentclass{article}
\usepackage[utf8]{inputenc}
\usepackage[T1]{fontenc}
\usepackage[portuguese]{babel}
\usepackage{tikz}
\usepackage{amssymb}

% Definição do símbolo branco e do marcador de início
\newcommand{\blank}{\square}
\newcommand{\lmarker}{\langle}

\usetikzlibrary{automata, positioning, arrows, shapes, calc}

\begin{document}

\begin{center}
% Título
\textbf{MT Decisora para $L = \{ w \in \{a,b\}^* \mid n_a(w) = n_b(w) \}$}
\vspace{1cm}

\begin{tikzpicture}[
    ->,                 
    >=stealth,          
    shorten >=1pt,      
    auto,               
    node distance=4.5cm,  % AUMENTADO: Distância base bem grande
    thick,              
    every edge quotes/.style={font=\footnotesize, align=center}, 
    state/.style={circle, draw, minimum size=1.5cm, thick, fill=white}, 
    accept/.style={double, circle, draw, minimum size=1.5cm, thick, fill=green!10},
    initial text=       
  ]

  % --- POSICIONAMENTO DOS ESTADOS (LAYOUT SUPER ESPAÇADO) ---
  
  \node[state, initial] (q0) {$q_{0}$};
  
  % Scanner afastado do início
  \node[state, right=3.5cm of q0] (qscan) {$q_{scan}$};
  
  % Aceitação bem à direita
  \node[accept, right=5.5cm of qscan] (qacc) {$q_{acc}$};
  
  % --- Buscadores ---
  % Usando coordenadas X e Y grandes para jogar os estados para os cantos
  % SINTAXE: above right = <distancia vertical> and <distancia horizontal>
  \node[state, above right=3.5cm and 2.5cm of qscan] (qbuscaB) {$q_{buscaB}$};
  \node[state, below right=3.5cm and 2.5cm of qscan] (qbuscaA) {$q_{buscaA}$};
  
  % --- Retorno (Rewind) ---
  \node[state, below left=3cm and 2cm of qscan] (qvol) {$q_{vol}$};


  % --- TRANSIÇÕES ---

  % 1. Inicialização
  \path (q0) edge node[above] {$\lmarker$ / $\lmarker$ D} (qscan);

  % 2. Scanner
  \path (qscan) 
    % Loop: Adicionado 'align=center' para corrigir o erro de LR mode
    edge[loop above, min distance=2cm, in=105, out=75] 
        node[above=1mm, align=center] {X / X D \\ Y / Y D} (qscan)
    
    % Vai buscar B
    edge[bend right=45] node[above left] {a / X D} (qbuscaB)
    
    % Vai buscar A
    edge[bend right=20] node[below left] {b / Y D} (qbuscaA)
    
    % Aceitação
    edge node[above] {$\blank$ / $\blank$ E} (qacc);

  % 3. Busca por B
  \path (qbuscaB) 
    % Loop: Adicionado 'align=center'
    edge[loop above, min distance=2.5cm, in=105, out=75] 
        node[above=1mm, align=center] {a / a D \\ X / X D \\ Y / Y D} (qbuscaB)
    
    % Retorno longo para qvol
    edge[bend right=60] node[left, pos=0.5, align=center] {b / Y E} (qvol);

  % 4. Busca por A
  \path (qbuscaA) 
    % Loop: Adicionado 'align=center'
    edge[loop below, min distance=2.5cm, in=255, out=285] 
        node[below=1mm, align=center] {b / b D \\ X / X D \\ Y / Y D} (qbuscaA)
    
    % Retorno longo para qvol
    edge[bend left=30] node[below, pos=0.5, align=center] {a / X E} (qvol);

  % 5. Retorno (Rewind)
  \path (qvol) 
    % Loop: Adicionado 'align=center'
    edge[loop left, min distance=2cm, in=165, out=195] 
        node[left=1mm, align=center] {a / a E \\ b / b E \\ X / X E \\ Y / Y E} (qvol)
        
    % Volta ao scan
    edge[bend right=20] node[left] {$\lmarker$ / $\lmarker$ D} (qscan);

  % --- LEGENDA ---
  \node[draw=black!50, dashed, rounded corners, inner sep=10pt, align=left, anchor=north west] 
  at (-1, -10) {
    \small
    \textbf{Estratégia da Máquina}\\
    1. $q_{scan}$: Lê o próximo não marcado.\\
    2. Se 'a', busca 'b'. Se 'b', busca 'a'.\\
    3. Marca ambos e retorna ($q_{vol}$).\\
    4. $q_{vol}$ volta até $\langle$ e reinicia.\\
    5. Aceita se fita vazia ($\blank$).\\
    Alfabeto: $\Sigma = \{a, b\}$ | Fita: $\Gamma = \{a, b, X, Y, \blank, \langle\}$
  };

\end{tikzpicture}
\end{center}

\end{document}